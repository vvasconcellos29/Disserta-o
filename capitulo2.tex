\chapter{\textit{Orthogonal Filter Bank Mutlicarrier} - OFDM} \label{capitulo2}

A descrição da forma de onda FBMC pode ser vista como uma generalização do OFDM, com menos restrições em relação ao formato de pulso utilizado para filtrar subportadoras. No contexto da 5G, entretanto, trata-se de algo mais robusto. O projeto PHYDAS (\textit{\textbf{PHY}sical layer for \textbf{DY}namic \textbf{A}cces\text{S} and cognitive radio}) \cite{phydas} se preocupou em criar um desenho bastante específico desta tecnologia, procurando construí-la de tal forma que diversas desvantagens do OFDM fossem superadas, abrindo portas para tornar as aplicações idealizadas para a 5G possíveis. É esta versão que será trazida aqui. 

\section{Arquitetura OFDM}\label{pulso} 

\subsection{F-OFDM}
\subsection{ZT-DS-OFDM}

\section{Efeitos do Canal de Multipercursos}

\section{Desempenho sob o efeito de não linearidades }
 



